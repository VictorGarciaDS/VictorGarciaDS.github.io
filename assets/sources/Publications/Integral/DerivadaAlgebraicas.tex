\subsection{Funciones algebraicas}\label{subsec:1.1}

A continuación se presentan las fórmulas de derivación de funciones algebraicas usando la notación de Leibniz.

Sean $u(x), v(x), w(x)$ funciones de $x$ derivables en todo su dominio, $c$ y $n$ constantes.
\Large
\begin{empheq}[box=\fbox]{align}
	\frac{d}{dx} c&=0\label{eqn:1.1}\\
	\frac{d}{dx} x&=1\label{eqn:1.2}\\
	\frac{d}{dx}cv(x)&=c\frac{d}{dx}v\label{eqn:1.3}\\
	\frac{d}{dx}(u+v-w)(x)&=\frac{d}{dx}u(x)+\frac{d}{dx}v(x)-\frac{d}{dx}w(x)\label{eqn:1.4}\\
	\frac{d}{dx}[v(x)]^n&=n[v(x)]^{n-1}\frac{d}{dx}v(x)\label{eqn:1.5}\\
	\frac{d}{dx}(uv)(x)&=u(x)\frac{d}{dx}v(x)+v(x)\frac{d}{dx}u(x)\label{eqn:1.6}\\
	\frac{d}{dx}\left(\frac{u}{v}\right)(x)&=\frac{v(x)\frac{d}{dx}u(x)-u(x)\frac{d}{dx}v(x)}{[v(x)]^2}\label{eqn:1.7}\\
	\frac{d}{dx}\left(\frac{c}{v(x)}\right)&=-\frac{c}{[v(x)]^2}\frac{d}{dx}v(x)\label{eqn:1.8}
\end{empheq}
\normalsize

\subsubsection{Ejercicios}
\begin{enumerate}
	\item Demostrar \cref{eqn:1.1,eqn:1.2,eqn:1.4,eqn:1.6} usando la definición de derivada, también conocida como la \textbf{Regla de los cuatro pasos}.
	\item Demostrar \cref{eqn:1.3,eqn:1.7,eqn:1.8} a partir de las fórmulas del ejercicio anterior.
	\item ** Demostrar \cref{eqn:1.5}. \textsl{[Hint: Para $n$ real, usar \textbf{sucesiones de funciones}]}.
	\item Demostrar la \textbf{Regla de la cadena} a partir de la definición:
	$$\frac{d}{dx}(f\circ g)(x)=\frac{d}{dx}f(g(x))\frac{d}{dx}g(x)$$
\end{enumerate}

Otra forma de demostrar \cref{eqn:1.5} requiere del uso de la Regla de la cadena y la fórmula de $\frac{d}{dx}e^x$. Aprovecharemos la siguiente sección para mencionar, además de ésta, las fórmulas de derivadas del resto de funciones trascendentes.
