\subsection{Cambio de variable o sustitución algebraica}\label{subsec3.1}
Uno de los ejemplos más evidentes de sustitución algebraica es el presentado en el \ref{problema5}, pues el paso crucial de integración se resolvió simplemente eligiendo $v(x)=a^x$ para su uso en una fórmula básica.

Tanto la sustitución algebraica como trigonométrica tienen una interpretación específica en la teoría de la medida, en especial al trabajar con integrales definidas, sin embargo, al menos para esta parte del texto, por simplicidad realizaremos ambos procesos de sustitución sin preocuparnos demasiado por estas implicaciones teóricas, procurando volver siempre a la variable original de interés.

Para mayor claridad, expresaremos tal idea al reescribir \cref{problema1} siguiendo el proceso de sustitución algebraica.
\begin{problema}
\begin{align*}
	\int tan\:x\:dx&=\int \frac{sen\:x}{cos\:x}\:dx\\
	&=-\int \frac{-sen\:x\:dx}{cos\:x}
\intertext{Sea $v=cos\:x$. Con tal elección de $v$, de la tabla \ref{tabla1} tenemos que $dv=-sen\:x\:dx$, así que podemos reescribir la integral anterior como}
				&=-\int \frac{dv}{v}
\intertext{Usando \cref{eqn:2.3.5}, obtenemos}
				&=-ln\:v+C
\intertext{Y como $v=cos\:x$, entonces}
				&=-ln(cos\:x)+C
\end{align*}
\end{problema}
\textbf{Una de las ideas más relevantes a destacar del ejercicio anterior y del proceso de integración en general, como ya se había mencionado en la \cref{subsec3.2}, es que $dv$ no se debe ``elegir a conveniencia'' para contar con todos los términos que aparecen en el integrando, $dv$ se determina al diferenciar el $v$ que se haya elegido.}

La simple idea de cambiar una expresión complicada por una sola variable algebraica nos abre la puerta a poder efectuar la integración de expresiones más generales, que de efectuarse correctamente, evita el error de ``elección por conveniencia'' antes mencionado.
\begin{problema}[$\int sen^nx\:cos\:x\:dx$]
	Si $n=-1$, tenemos $\int cot\:x\:dx$, la cual se abordó en \cref{problema2}.
	
	Si $n\neq -1$, para efectuar la integral, definamos $v=sen\:x$, además, de acuerdo a la tabla \ref{tabla1}, $dv=cos\:x\:dx$, así que es posible reescribir la integral como
	\begin{align*}
		\int sen^nx\:cos\:x\:dx&=\int v^n dv
		\intertext{Como $n\neq -1$, podemos utilizar \cref{eqn:2.3.4}, así que}
		&=\frac{v^{n+1}}{n+1}+C
		\intertext{Y como $v=sen\:x$, entonces}
		&=\frac{sen^{n+1}x}{n+1}+C
	\end{align*}
	Por lo anterior,
	$$\int sen^nx\:cos\:x\:dx=\left\{\begin{matrix}ln(sen\:v)+C,&\text{ si }n=-1\\ \frac{sen^{n+1}x}{n+1}+C,&\text{ si }n\neq-1.\end{matrix}\right.$$
\end{problema}
Podemos notar en los problemas anteriores que la elección de $v$ tuvo como consecuencia la aparición de forma natural de $dv$ en la integral, lo cual permitió convertir las integrales en integrales inmediatas, beneficio que no se tiene en todas las integrales, para ilustrarlo sírvase de ejemplo \cref{problema5}.
\begin{problema}[$\int a^x\:dx$]\label{problema8}
Sea $u=a^x$, con lo que la tabla \ref{tabla1}, $du=ln\:a\:a^x\:dx$, así que podemos reescribir la integral como
\begin{align*}
	\int a^x\:dx&=\int \frac{du}{ln\:a}
	\intertext{Usando \cref{eqn:2.3.2} con $c=\frac{1}{ln\:a}$ y $v=1$, obtenemos}
			&=\frac{1}{ln\:a}\int du
	\intertext{Por el Teorema Fundamental del Cálculo,}
			&=\frac{u}{ln\:a}+C
	\intertext{Y como $u=a^x$,}
			&=\frac{a^x}{ln\:a}+C
\end{align*}
\end{problema}
En contraste con los problemas previos en los que solo se requirió del cambio de variable para determinar cual integral inmediata permitía terminar el problema, en este caso, la aparición del factor constante $\frac{1}{ln\:a}$ requirió de un pequeño esfuerzo adicional: la manipulación de constantes dentro y fuera de la integral auxiliados de \cref{eqn:2.3.2}.

El siguiente problema se presenta para reafirmar la importancia de determinar $dv$ a partir de $v$, sin verse influenciados por la forma de la integral que se está realizando.
\begin{problema}[$\int(4x^3+6x)^{\pi-1} (2x^2+1)dx$]\label{problema9}
	Comencemos notando que la aparición del exponente $\pi-1$ sugiere el uso de \cref{eqn:2.3.4}, en la que la elección de $v$ es claramente $v=4x^3+6x$ junto a $n=\pi-1$.
	
	Como antes se mencionó, un error común es asumir que el resto de elementos dentro de la integral, $(2x^2+1)dx$ constituyen a $dv$, lo cual facilitaría claramente la integración, aunque esto sería incorrecto, pues $dv$ se debe obtener, por definición, al diferenciar $v$, así que $dv=12x^2+6=6(2x^2+1)dx$, de donde podemos observar que es posible reescribir la integral como
	\begin{align*}
		\int(4x^3+6x)^{\pi-1} (2x^2+1)dx&=\int v^n \frac{dv}{6}
		\intertext{Usando \cref{eqn:2.3.2} seguida de \cref{eqn:2.3.4},}
									&=\frac{1}{6}\int v^n dv\\
									&=\frac{1}{6}\frac{v^{n+1}}{n+1}+C\\
									&=\frac{v^{n+1}}{6(n+1)}+C
		\intertext{Finalmente, como $v=4x^3+6x$ y $n=\pi-1$, entonces}
		\int(4x^3+6x)^{\pi-1} (2x^2+1)dx&=\frac{(4x^3+6x)^{\pi}}{6\pi}+C
	\end{align*}
\end{problema}

\subsubsection{Ejercicios}
\begin{align}
	\int \frac{6z\:dz}{(5-3z^2)^2}&=\frac{8\sqrt{x^3+8}}{3}+C\\
	\int \sqrt x \left(\sqrt{a}-\sqrt{x}\right)^2dx&=\frac{2}{3}ax^{\frac{3}{2}}-x^2\sqrt{a}+\frac{2x^{\frac{5}{2}}}{5}+C\\
	\int \left(\frac{sec\:x}{1+tan\:x}\right)^2dx&=-\frac{1}{1+tan\:x}+C\\
	\int \frac{e^\theta\:d\theta}{a+be^\theta}&=\frac{ln(a+be^\theta)}{b}+C\\
	\int x^{n-1}\sqrt{a+bx^n}\:dx&=\frac{2(a+bx^n)^{\frac{3}{2}}}{3bn}+C\\
	\int \frac{x^2\:dx}{2+x^3}&=\frac{ln(2+x^3)}{3}+C
\end{align}

\subsubsection{Problemas}
\begin{align}
	\int \frac{sec^2\:y\:dy}{a+b\:tan\:y}\\
	\int \frac{ae^\theta+b}{ae^\theta-b}d\theta\\
	\int \frac{x\:dx}{(a+bx^2)^3}\\
	\int \frac{x^2\:dx}{(a+bx^3)^2}\\
	\int \frac{sen\:a\theta\:d\theta}{cos\:a\theta+b}\\
	\int \frac{csc^2\phi\:d\phi}{\sqrt{2cot\phi+3}}\\
	\int \frac{sec\:2\theta\: tan\:2\theta\: d\theta}{3sec\:2\theta-2}\\
	\int 6 e^{3x}dx\\
	\int (e^{5x}+a^{5x})dx\\
	\int a^{x}e^x\:dx
\end{align}