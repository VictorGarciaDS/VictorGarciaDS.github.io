\subsection{Descomposición en fracciones parciales}
Cuando la integral es de la forma $\int \frac{R(x)}{Q(x)}dx$, donde $R(x)$ y $Q(x)$ son ambos polinomios, el primer paso a realizarse de manera natural, es simplificar la fracción mediante el algoritmo de la división, es decir, elegir $S(x)$ y $P(x)$ tales que $S(x)=\frac{R(x)-P(x)}{Q(x)}$ es un polinomio, posiblemente constante, y que $P(x)$ es de grado menor que $Q(x)$, de esta forma la integral queda de la forma
\begin{align*}
	\int \frac{R(x)}{Q(x)}dx&=\int \left(S(x)+\frac{P(x)}{Q(x)}\right)dx\\
						&=\int S(x)\:dx+\int\frac{P(x)}{Q(x)}dx
\end{align*}
Como $S(x)$ es un polinomio, es fácilmente integrable usando solamente las fórmulas \ref{eqn:2.3.2} a \ref{eqn:2.3.4}.

Un método para efectuar algunos casos de la integral $\int \frac{P(x)}{Q(x)}$ es la llamada descomposición de fracciones parciales, que aunque solía ser un tema puramente algebraico, ya no es tan estudiado y se explica a continuación.

Comencemos explicando de manera ilustrativa uno de los casos:

Supongamos que al factorizar $Q(x)$, todos sus factores son lineales y distintos, es decir $Q(x)=\Pi_{i}(a_1x+b_1)$, tales que si $i\neq j$, entonces $(a_i-a_j)x+(b_i-b_j)\neq0$. Es posible elegir $A_1,A_2,\ldots,A_n\in\mathbb R$ tales que
\begin{align*}
	\frac{P(x)}{Q(x)}&=\sum_{i}\frac{A_i}{a_ix+b_i}
 	 \intertext{Al efectuar explícitamente la suma e igualar los coeficientes de los polinomios de los numeradores, se obtiene un sistema lineal de ecuaciones para determinar los coeficientes $A_i$, para posteriormente usar \cref{eqn:2.3.3} para descomponer la integral en varias integrales más sencillas de la siguiente forma:}
 	 \int 	\frac{P(x)}{Q(x)}dx&=\int\sum_{i}\frac{A_i}{a_ix+b_i}dx\\
 	 					&=\sum_{i}\int\frac{A_i}{a_ix+b_i}dx
\end{align*}
Para no quedar solo con coeficientes, vale la pena efectuar una integral de este tipo para terminar de entender la utilidad del método:
\begin{problema}[$\int \frac{x^3+8x^2+22x+29}{x^2+8x+15}dx$]
	Comencemos notando que $\frac{2x^3+16x^2+37x+29}{x^2+8x+15}=2x+\frac{7x+29}{x^2+8x+15}$, así que
	\begin{align}
		\int \frac{x^3+8x^2+22x+29}{x^2+8x+15}dx&=\int \left(2x+\frac{7x+29}{x^2+8x+15}\right)dx\nonumber\\
		&=2\int x\:dx+\int \frac{7x+29}{x^2+8x+15}dx\nonumber\\
		&=x^2+\int \frac{7x+29}{x^2+8x+15}dx\label{18a}
		\intertext{Notando que $x^2+8x+15=(x+5)(x+3)$, vemos que}
		\frac{7x+29}{x^2+8x+15}&=\frac{A_1}{x+5}+\frac{A_2}{x+3}\nonumber\\
							&=\frac{A_1(x+3)+A_2(x+5)}{(x+5)(x+3)}\nonumber\\
							&=\frac{(A_1+A_2)x+(3A_1+5A_2)}{x^2+8x+15}\nonumber
		\intertext{De lo anterior tenemos que $A_1+A_2=7$ y $3A_1+5A_2=29$. La solución de tal sistema es $A_1=3,A_2=4$, así que}
		\frac{7x+29}{x^2+8x+15}&=\frac{3}{x+5}+\frac{4}{x+3}\label{18b}
		\intertext{Al sustituir \ref{18b} en \ref{18a}, obtenemos}
		\int \frac{x^3+8x^2+22x+29}{x^2+8x+15}dx&=x^2+\int\left(\frac{3}{x+5}+\frac{4}{x+3}\right)dx\nonumber\\
		&=x^2+3\int\frac{dx}{x+5}+4\int\frac{dx}{x+3}\nonumber\\
		&=x^2+3ln(x+5)+4ln(x+3)+C\nonumber\\
		&=x^2+ln\left[(x+5)^3(x+3)^4\right]+C\nonumber
	\end{align}
\end{problema}
La tabla \ref{tabla4} permite ver el tipo de descomposición en fracciones parciales correspondiente a algunos los casos de factorización de $Q(x)$ más estudiados.
\begin{center}
	\begin{tabular}{|c|c |c |c|}
		\hline
		Caso&Factorización&Descripción&Descomposición\\
		&de $Q(x)$&&correspondiente\\
	\hline
		&$\Pi_{i=1}^n(a_ix+b_i)$	&Factores	&\\
		I&t.q. si $i\neq j$, entonces	&lineales	&$\sum_{i}\frac{A_i}{a_ix+b_i}$\\
		&$(a_i-a_j)x+(b_i-b_j)\neq0$	&distintos	&\\
		\hline
		&&Potencia de&\\ II&$(ax+b)^n$&un factor &$\sum_{i=1}^n\frac{B_i}{(ax+b)^i}$\\&&lineal&\\
		\hline
		&$\Pi_{i=1}^n(a_ix^2+b_i+c_i)$	&Factores		&\\
		II&t.q. si $i\neq j$, entonces	&cuadráticos	&$\sum_{i}\frac{C_ix+D_i}{a_ix^2+b_ix+c_i}$\\
		III&$(a_i-a_j)x^2+(b_i-b_j)x$&distintos		&\\&$+(c_i-c_j)\neq0$&&\\
		\hline
		&&Potencia de&\\ IV&$(ax^2+bx+c)^n$&un factor &$\sum_{i=1}^n\frac{E_ix+F_i}{(ax^2+bx+c)^i}$\\&&cuadrático&\\
		\hline
	\end{tabular}\label{tabla4}
\end{center}
La tabla \ref{tabla4} sirve para obtener descomposiciones en fracciones parciales de combinaciones de los distintos casos, como se muestra en los siguientes ejemplos:
\begin{problema}[$\frac{4y^2-8}{y^3+2y^2}$]
	Notemos que $y^3-2y^2=y^2(y+2)$, así que requerimos utilizar los casos I y III de la tabla \ref{tabla4}. Sean $B_1,B_2,A_1$ tales que
	\begin{align*}
		\frac{4y^2-8}{y^3+2y^2}&=\frac{B_1}{y}+\frac{B_2}{y^2}+\frac{A_1}{y+2}\\
							&=\frac{B_1y(y+2)+B_2(y+2)+A_1y^2}{y^2(y+2)}\\
							&=\frac{(A_1+B_1)y^2+(2B_1+B_2)y+2B_2}{y^3+2y^2}.
		\intertext{Al igualar los coeficientes de los numeradores, obtenemos el sistema $A_1+B_1=4$, $2B_1+B_2=0$ y $2B_2=-8$, con soluciones $A_1=2,B_1=2,B_2=-4$, así que la integral pasa a tener la forma}
		\int \frac{4y^2-8}{y^3+2y^2} dy&=\int \left(\frac{2}{y}-\frac{4}{y^2}+\frac{2}{y+2}\right)dy\\
								&=2\int\frac{dy}{y}-4\int y^{-2}\:dy+2\int\frac{dy}{y+2}\\
								&=2ln\:y-4\frac{y^{-1}}{-1}+2ln(y+2)+C\\
								&=ln\left[y^2(y+2)^2\right]+\frac{4}{y}+C.
	\end{align*}
\end{problema}
\begin{problema}[$\int\frac{4x^2+3x+8}{x^5+4x^3+4x}dx$]
	Notemos que $x^5+4x^2+4x=x(x^2+2)^2$, así que se usan los casos I y IV de la tabla \ref{tabla4}. Sean $A_1, E_1,E_2,F_1$ y $F_2$ tales que
	\begin{align*}
	\frac{4x^2+3x+8}{x^5+4x^3+4x}&=\frac{A_1}{x}+\frac{E_1x+F_1}{x^2+2}+\frac{E_2x+F_2}{(x^2+2)^2}\\
					&=\frac{A_1(x^2+2)^2+(E_1x+F_1)x(x^2+2)+(E_2x+F_2)x}{x(x^2+2)^2}\\
					&=\frac{(A_1+E_1)x^4+F_1x^3+(4A_1+2E_1+E_2)x^2+(2F_1+F_2)x+4A_1}{x^5+4x^3+4x}
					\intertext{Al igualar los coeficientes de los numeradores, obtenemos el sistema $$A_1+E_1=0,F_1=0,4A_1+2E_1+E_2=4,2F_1+F_2=3,4A_1=8,$$cuya solución es $A_1=2,E_1=-2,E_2=0,F_1=0,F_2=3$, así que podemos reescribir la integral como}
	\int \frac{4x^2+3x+8}{x^5+4x^3+4x}dx&=\int\left(\frac{2}{x}+\frac{-2x}{x^2+2}+\frac{3}{(x^2+2)^2}\right)\\
	&=2\int\frac{dx}{x}-\int \frac{2x\:dx}{x^2+2}+3\int \frac{dx}{(x^2+2)^2}\\
	&=2ln\:x-ln(x^2+2)+3\int  \frac{dx}{\left(x^2+\sqrt{2}^2\right)^2}
	\intertext{Notemos que en la integral restante podemos usar la sustitución trigonométrica $x=\sqrt2\:tan\:\theta$ de la tabla \ref{tabla2}, así que $dx=\sqrt 2\:sec^2\:\theta\:d\theta$, luego}
	&=2ln\:x-ln(x^2+2)+3\int  \frac{\sqrt{2}sec^2\:\theta\:d\theta}{\left(2tan^2\:\theta+2\right)^2}\\
	&=2ln\:x-ln(x^2+2)+3\int  \frac{\sqrt{2}sec^2\:\theta\:d\theta}{\left[\left(2\right)\left(tan^2\:\theta+1\right)\right]^2}\\
	&=2ln\:x-ln(x^2+2)+3\int  \frac{\sqrt{2}sec^2\:\theta\:d\theta}{\left[2sec^2\:\theta\right]^2}\\
	&=2ln\:x-ln(x^2+2)+3\int  \frac{\sqrt{2}sec^2\:\theta\:d\theta}{4sec^4\:\theta}\\
	&=2ln\:x-ln(x^2+2)+\frac{3\sqrt 2}{4}\int  \frac{d\theta}{sec^2\:\theta}\\
	&=2ln\:x-ln(x^2+2)+\frac{3\sqrt 2}{4}\int  cos^2\:\theta\:d\theta
	\intertext{Ya que $cos^2\theta=\frac{cos\:2\theta+1}{2}$, entonces\footnote{Para un procedimiento alterno, véase \cref{problema3.3.2.1}}}
	&=2ln\:x-ln(x^2+2)+\frac{3\sqrt 2}{4}\int \frac{cos\:2\theta+1}{2} d\theta\\
	&=2ln\:x-ln(x^2+2)+\frac{3\sqrt 2}{8}\int cos\:2\theta d\theta+\frac{3\sqrt2}{8}\int d\theta\\
	&=ln\left(\frac{x^2}{x^2+2}\right)+\frac{3\sqrt 2}{16}\int 2cos\:2\theta d\theta+\frac{3\sqrt2}{8}\int d\theta\\
	&=ln\left(\frac{x^2}{x^2+2}\right)+\frac{3\sqrt 2}{16}sen\:2\theta+\frac{3\sqrt2}{8}\theta+C
	\intertext{Además, como $sen\:2\theta=2sen\:\theta\:cos\:\theta$, entonces}
	&=ln\left(\frac{x^2}{x^2+2}\right)+\frac{3\sqrt 2}{8}sen\:\theta\:cos\:\theta+\frac{3\sqrt2}{8}\theta+C
	\intertext{Finalmente, usando la tabla \ref{tabla2}, podemos notar que $sen\theta=\frac{v}{\sqrt{a^2+v^2}}$, $cos\theta=\frac{a}{\sqrt{a^2+v^2}}$ y $\theta=arc\:tan\:\frac{v}{a}$, así que}
	&=ln\left(\frac{x^2}{x^2+2}\right)+\frac{3\sqrt 2}{8}\frac{x}{\sqrt{x^2+2}}\frac{\sqrt 2}{\sqrt{x^2+2}}+\frac{3\sqrt2}{8}arc\:tan\:\frac{x}{\sqrt{2}}+C\\
	&=ln\left(\frac{x^2}{x^2+2}\right)+\frac{3x}{4x^2+8}+\frac{3\sqrt2}{8}arc\:tan\:\frac{x}{\sqrt{2}}+C
\end{align*}
\end{problema}
Aunque parece que el proceso se puede extender de manera intuitiva para aquellos casos en que $Q(x)$ se puede factorizar hasta polinomios de grado mayor a $2$, los sumandos de la descomposición no necesariamente se pueden integrar en los reales, tal es el caso de
$$\int\frac{x+5}{x^3-2x^2+x+7}dx$$
cuya expresión explícita requiere de temas de variable compleja para ser expresada.

\subsubsection{Ejercicios}

\subsubsection{Problemas}