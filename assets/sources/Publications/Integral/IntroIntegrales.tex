\section{Integración de formas elementales ordinarias}

De acuerdo a \cite{aguilar_matematicas_2009} y , en los problemas de cálculo integral nos interesa:

\textsl{Hallar una función $f(x)$ cuya derivada}
$$f'(x)=\phi(x)$$
\textsl{es conocida.}

Nos concentraremos en aprender a hallar las integrales \textsl{indefinidas} de expresiones dadas.

En lo que sigue daremos por sentado que todas nuestras funciones de interés poseen una integral indefinida, el estudio riguroso de la integrabilidad de las funciones queda fuera del propósito de estas notas. Sin embargo seguirá teniendo importancia la \textsl{constante de integración} debido a su relevancia en cursos como ecuaciones diferenciales.
%231

Tampoco entraremos a detalle en la teoría de formas diferenciales, sin embargo, es importante, al expresar una integral no olvidar el $dx$, ya que el restarle importancia puede llevarnos a cometer errores como el siguiente:

Al resolver la integral $\int\sen(x^2)$, ``se puede realizar'' la sustitución $u=x^2$, con lo que $$\int\sen(u)=-\cos(u)=\cos(x^2)$$

Así que a grandes rasgos, diremos que la integral es una ``\textsl{medida}'' y $dx$ nos habla del tamaño de la unidad de medida.
%mencionar la integral de rieman dF(x)...
\subsection{Integrales inmediatas}
Con la finalidad de coincidir en la notación con los principales libros de referencia, en ésta subsección nos referiremos a $v(x)$, por simplicidad como $v$, siendo así como, por ejemplo, \cref{eqn:2.3.5}, se escribiría correctamente como
$$\int \frac{dv(x)}{v(x)}=ln(v(x))+C$$
Además de permitirnos usar indistintamente $v$ como variable o como función de $x$. Dicho lo anterior, haremos uso del Teorema Fundamental del Cálculo y las fórmulas \ref{eqn:1.3.2} a \ref{eqn:1.3.6}, \ref{eqn:2.13} y \ref{eqn:2.15} para obtener las siguientes fórmulas de integrales:

\begin{empheq}[box=\fbox]{align}
	\int 1 dx=\int dx&=x+C\label{eqn:2.3.1}\\
	\int cvdx&=c\int vdx+C\label{eqn:2.3.2}\\
	\int (u+v-w)dx&=\int udx+\int vdx-\int wdx\label{eqn:2.3.3}\\
	\int v^ndv&=\frac{v^{n+1}}{n+1}+C\text{ Si }n\neq -1\label{eqn:2.3.4}\\
	\int \frac{dv}{v}&=ln(v)+C\label{eqn:2.3.5}\\
	\int e^{v}dv&=e^{v}+C\label{eqn:2.3.6}
\end{empheq}
Un primer uso de \cref{eqn:2.3.4} es generalizar \cref{eqn:2.3.1}, ésto al tomar $n=0$, de manera que
$$\int dv=v+C$$
como $v$ es una función ($v(x)$), podemos ver la ecuación anterior como una versión del Teorema Fundamental del Cálculo.

A su vez, de las fórmulas \ref{eqn:2.2.1} a \ref{eqn:2.2.6} se obtienen las siguientes:
\begin{empheq}[box=\fbox]{align}
	\int \cos v dv&=\sen v+C\label{eqn:2.3.7}\\
	\int\sen v dv&=-\cos v+C\label{eqn:2.3.8}\\
	\int \sec^2 v\:dv&=\tan v+C\label{eqn:2.3.9}\\
	\int \csc^2 v\:dv&=-\cot v+C\label{eqn:2.3.10}\\
	\int \sec v\tan v dv&=\sec v+C\label{eqn:2.3.11}\\
	\int \csc v\cot vdv&=-\csc v+C\label{eqn:2.3.12}
\end{empheq}
Podemos notar que el caso que no es cubierto por \cref{eqn:2.3.4}, $n=-1$, puede resolverse mediante \cref{eqn:2.3.5}.

Las fórmulas \ref{eqn:2.2.7} a \ref{eqn:2.2.12} también nos brindan integrales inmediatas, pero éstas convienen ser estudiadas en el tema de Integración por Sustitución Trigonométrica o cambio de variable.


\subsection{Algunas integrales casi inmediatas}
A continuación veremos otras integrales que requieren de un esfuerzo ligeramente mayor a la simple observación de la fórmula y la correcta elección de la función $v(x)$ a integrar.
\begin{problema}[$\int tan\:v\:dv$]\label{problema1}
Comencemos utilizando la siguiente identidad trigonométrica:
\begin{align}\label{eqn:2.4.1}
	\int tan\:x\:dx&=\int \frac{sen\:x}{cos\:x}\:dx\nonumber\\
	&=-\int \frac{-sen\:x\:dx}{cos\:x}\nonumber
\intertext{Podemos ver que al sustituir $v=x$ en \cref{eqn:2.2.2} obtenemos el numerador de la expresión anterior, así que podemos reescribir lo anterior como}
				&=-\int \frac{d(cos\:x)}{cos\:x}\nonumber
			\intertext{Al usar \cref{eqn:2.3.5} con $v(x)=cos\:x$, obtenemos}
				&=-ln(cos\:x)+C\nonumber\\
	\int tan\:v\:dv&=-ln(cos\:v)+C
\end{align}
\end{problema}
\begin{problema}[$\int cot\:v\:dv$]\label{problema2}
La demostración es análoga a la integral \ref{eqn:2.4.1}
\begin{align}
\int cot\:v\:dv=ln(sen\:v)+C
\end{align}
\end{problema}
\begin{problema}[$\int sec\:v\:dv$]\label{problema3}
Al observar las fórmulas \ref{eqn:2.2.1} a \ref{eqn:2.2.6}, podemos notar que no hay expresión trigonométrica, cuya diferencial sea $sec\:v\:dv$, sin embargo, \cref{eqn:2.2.3,eqn:2.2.5} tienen expresiones parecidas, así que multiplicamos y dividimos la expresión por $sec\:x+tan\:x$:
\begin{align}
\int sec\:x\:dx&=\int \frac{sec\:x+tan\:x}{sec\:x+tan\:x} sec\:x\:dx\nonumber\\
			&=\int \frac{(sec\:x+tan\:x)sec\:x\:dx}{sec\:x+tan\:x}\nonumber\\
			&=\int \frac{(sec^2\:x+tan\:x\:sec\:x)\:dx}{sec\:x+tan\:x}\nonumber
			\intertext{Al reordenar los sumandos del denominador}
			&=\int \frac{(sec^2\:x+tan\:x\:sec\:x)\:dx}{tan\:x+sec\:x},\nonumber
			\intertext{Al sustituir en \cref{eqn:1.3.4} $u=tan\:x$, $v=sec\:x$ y $w=0$, además de \cref{eqn:2.2.3,eqn:2.2.5}, antes mencionadas, ambas con $v=x$, podemos reescribir el numerador de la forma}
			&=\int \frac{d(tan\:x+\:sec\:x)}{tan\:x+sec\:x}\nonumber\\
			\intertext{Finalmente al usar \cref{eqn:2.3.5} con $v(x)=tan\:x+sec\:x$, obtenemos}
			&=ln(tan\:x+sec\:x)+C\nonumber\\
\label{eqn:2.4.3}
\int sec\:v\:dv&=ln(tan\:v+sec\:v)+C
\end{align}
\end{problema}
\begin{problema}[$\int csc\:v\:dv$]
La demostración es análoga a la integral \ref{eqn:2.4.3}
\begin{align}
\int csc\:v\:dv=-ln(cot\:v+csc\:v)+C
\end{align}
\end{problema}
\begin{problema}[$\int a^v\:dv$]\label{problema5}
El caso $a=1$ se ve en \cref{eqn:2.3.1}, así que supondremos $a\neq 1$.
Multiplicamos y dividimos por $\ln a$,
\begin{align}
	\int a^x\:dx&=\int \frac{\ln a}{\ln a} a^x\:dx\nonumber\\
			&=\int \frac{1}{\ln a} \ln a\cdot a^x\:dx\nonumber
\intertext{Al utilizar \cref{eqn:2.3.2} con $c=\frac{1}{\ln a}$ y $v=\ln a\cdot a^x$, obtenemos}
			&=\frac{1}{\ln a}\int \ln a\cdot a^x\:dx\nonumber
\intertext{Al usar \cref{eqn:2.2.16} con $v=x$, obtenemos}
			&=\frac{1}{\ln a}\int d(a^x)\nonumber
\intertext{Así que usar el Teorema Fundamental del Cálculo, antes enunciado como la versión general de \cref{eqn:2.3.1} con $v(x)=a^x$, concluimos que}
			&=\frac{1}{\ln a}a^x+C\nonumber\\
	\int a^v\:dv&=\frac{1}{\ln a}a^v+C
\end{align}
\end{problema}

\subsection{La importancia de la elección de $dv$}\label{subsec3.2}
Un error muy común cuando se está aprendiendo a integrar, consiste en creer que tras elegir $v$, el resto de términos conforman $dv$, para notar que no es así, basta con observar que para el siguiente problema, derivado del \cref{problema5}, es fácil cometer un error.
\begin{problema}[$\int 5\: ln\:a\: a^x\:dx$]\label{problema6}
	Al tomar $v=a^x$, no debemos afirmar tan a la ligera que $dv$ es $5\:ln\:a\:dx$, pues $dv$ se determina al \textbf{diferenciar} $v$, asistidos de las fórmulas de la sección \ref{subsec:2.1.1}, no simplemente tomando el resto de términos en la expresión. Como se vio en el \cref{problema5}, $dv=ln\:a\:a^x$, así que no es posible usar directamente \cref{eqn:2.3.1} con $v(x)=a^x$, así como en el \cref{problema5} tampoco se usó desde el primer paso.
	
	Comencemos entonces usando \cref{eqn:2.3.2} con $c=5$ y $v=ln\:a\:a^x$, de manera que
	\begin{align*}
		\int 5\: ln\:a\: a^x\:dx&=5\int ln\:a\: a^x\:dx
		\intertext{Donde ya podemos usar \cref{eqn:2.3.1} con $v(x)=a^x$}
								&=5a^x+C
	\end{align*}
\end{problema}
En ocasiones una constante será lo único que impida usar directamente alguna fórmula y como se vio en el \cref{problema6}, la manera de tratar con este impedimento es asistidos de \cref{eqn:2.3.2}, sin embargo no siempre será así y serán requeridas otras estrategias, como las expuestas en la \cref{sec4}.

Una buena estrategia para asegurarse de no usar incorrectamente una fórmula es cubrir la integral que se está realizando para determinar $dv$ a partir de la elección de $v$ y no a partir de la integral que se está resolviendo.

\subsection{Ejercicios}
\begin{align}
	\int 2x(2+x^2)^{\frac{3}{2}}dx&=\frac{2(2+x^2)^{\frac{5}{2}}}{5}+C\\
	\int \sqrt{m+nx} dx&=\frac{2(m+nx)^{\frac{3}{2}}}{3n}+C\\
	\int \frac{e^{\theta} d\theta}{c+ae^{\theta}}&=\frac{1}{a}\ln |c+ae^{\theta}|+C\\
	\int \frac{\sen 5x\:dx}{1-\cos 5x}&=\frac{1}{5}\ln|1-\cos 5x|+C\\
	\int \frac{dx}{e^{2x}}&=\frac{1}{2e^{2x}}+C\label{ejercicio2.5.1}\\
	\int \frac{\tan\sqrt{x}}{\sqrt x}&=2\ln|\sec{\sqrt x}|+C
\end{align}
\subsection{Resumen de las fórmulas}
\begin{tabular}{| L | L | L | L|}
\hline
\text{Función} & \text{Derivada} & \text{Diferencial}& \text{Integral}\\
\hline
f(x)&\frac{d}{dx}f(x)&d\:f(x)&\int f(x)\:dx\\
\hline
c & 0 &0&cx+C\\
x&1&dx&\frac{x^2}{2}+C\\
cv(x)&c\frac{d}{dx}v(x)&c\frac{d}{dx}v(x)dx&c\int v(x) dx\\
(u+v-w)(x)&\frac{d}{dx}u(x)+\frac{d}{dx}v(x)-\frac{d}{dx}w(x)&du(x)+dv(x)-dw(x)&\int u dx+\int v dx-\int w dx\\
\left[v(x)\right]^n&n\left[v(x)\right]^{n-1}\frac{d}{dx}v(x)&n\left[v(x)\right]^{n-1}dv(x)&\\
x^n&nx^{n-1}&nx^{n-1}dx&\left\{\begin{matrix}\frac{x^{n+1}}{n+1}+C\text{ si }n\neq-1\\ ln(x)+C\text{ si }n=-1\end{matrix}\right.\\
u(x)v(x)&u(x)\frac{d}{dx}v(x)+v(x)\frac{d}{dx}u(x)&udv(x)+vdu(x)&\\
\left(\frac{u}{v}\right)(x)&\frac{v(x)\frac{d}{dx}u(x)-u(x)\frac{d}{dx}v(x)}{\left[v(x)\right]^2}&\frac{v(x)du(x)-u(x)dv(x)}{\left[v(x)\right]^2}&\\
\frac{c}{v(x)}&-\frac{c}{\left[v(x)\right]^2}\frac{d}{dx}v(x)&-\frac{c}{\left[v(x)\right]^2}dv(x)&\\
\hline
sen\:v(x)&cos\:v(x)\frac{d}{dx}v(x)&cos\:v(x)dv(x)&\\
sen\:x&cos\:x&cos\:x\:dx&-cos\:x+C\\
cos\:v(x)&-sen\:v(x)\frac{d}{dx}v(x)&-sen\:v(x)dv(x)&\\
cos\:x&-sen\:x&-sen\:x\:dx&sen\:x+C\\
tan\:v(x)&\left[sec\:v(x)\right]^2\frac{d}{dx}v(x)&\left[sec\:v(x)\right]^2dv(x)&\\
tan\:x&\left[sec\:x\right]^2&\left[sec\:x\right]^2dx&-ln(cos\:x)+C\\
cot\:v(x)&-\left[csc\:v(x)\right]^2\frac{d}{dx}v(x)&-\left[csc\:v(x)\right]^2dv(x)&\\
cot\:x&-\left[csc\:x\right]^2&-\left[csc\:x\right]^2dx&ln(sen\:v)+C\\
sec\:v(x)&sec\:v(x)tan\:v(x)\frac{d}{dx}v(x)&sec\:v(x)tan\:v(x)dv(x)&\\
sec\:x&sec\:x\:tan\:x&sec\:x\:tan\:x\:dv(x)&ln(tan\:x+sec\:x)+C\\
csc\:v(x)&-csc\:v(x)cot\:v(x)\frac{d}{dx}v(x)&-csc\:v(x)cot\:v(x)dv(x)&\\
csc\:x&-csc\:x\:cot\:x&-csc\:x\:cot\:x\:dv(x)&-ln(cot\:x+csc\:x)+C\\
\hline
\arcsen\:v(x)&\frac{1}{\sqrt{1-[v(x)]^2}}\frac{d}{dx}v(x)&\frac{1}{\sqrt{1-[v(x)]^2}}d\:v(x)&\\
\arccos\:v(x)&-\frac{1}{\sqrt{1-[v(x)]^2}}\frac{d}{dx}v(x)&-\frac{1}{\sqrt{1-[v(x)]^2}}d\:v(x)&\\
\arctan\:v(x)&\frac{1}{1+[v(x)]^2}\frac{d}{dx}v(x)&\frac{1}{1+[v(x)]^2}d\:v(x)&\\
\arccot\:v(x)&-\frac{1}{1+[v(x)]^2}\frac{d}{dx}v(x)&-\frac{1}{1+[v(x)]^2}d\:v(x)&\\
\arcsec\:v(x)&\frac{1}{v(x)\sqrt{[v(x)]^2-1}}\frac{d}{dx}v(x)&\frac{1}{v(x)\sqrt{[v(x)]^2-1}}d\:v(x)&\\
\arccsc\:v(x)&-\frac{1}{v(x)\sqrt{[v(x)]^2-1}}\frac{d}{dx}v(x)&-\frac{1}{v(x)\sqrt{[v(x)]^2-1}}d\:v(x)&\\
\hline
ln(v(x))&\frac{1}{v(x)}\frac{d}{dx}v(x)&\frac{1}{v(x)}d\:v(x)&\\
log_bv&\frac{log_be}{v(x)}\frac{d}{dx}v(x)&\frac{log_be}{v(x)}d\:v(x)&\\
\hline
e^{v(x)}&e^{v(x)}\frac{d}{dx}v(x)&e^{v(x)}d\:v(x)&\\
e^{x}&e^{x}&e^{x}dx&e^x\\
a^{v(x)}&\ln\:a\:a^{v(x)}\frac{d}{dx}v(x)&\ln\:a\:a^{v(x)}d\:v(x)&\\
a^{x}&\ln\:a\:a^{x}&\ln\:a\:a^{x}\:dx&\frac{1}{\ln a}a^x+C\\
\left[u(x)\right]^{v(x)}&v\cdot[u]^{v-1}\frac{d}{dx}u+\ln u\cdot u^{v}\frac{d}{dx}v&v\cdot[u]^{v-1}d\:u+\ln u\cdot u^{v}d\:v&\\
\hline
\end{tabular}\label{tabla1}
En el formulario anterior, hay expresiones cuyas integrales no expresamos, esto por ser demasiado generales, o bien, porque aún no abordamos los métodos para obtenerlas.
\subsection{Problemas}
\begin{align}
	&\int\left(\frac{4}{\sqrt[3]{x}}-\frac{5}{\sqrt[4]{x}}\right)dx\\
	&\int \cot x(2+\ln|\sec x|)dx\\
	&\int \frac{dy}{y\ln^2 y}\\
	&\int \sqrt{1+\cos \alpha}d\alpha\\
	&\int e^{\frac{1}{\sec 2x}}\sen 2x\: dx\\
	&\int (10^{3x}-2^x)dx\\
	&\int \frac{e^{\arctan x}}{1+x^2}dx\\
	&\int x\csc 4x^2 dx\\
	&\int e^{2x}\sen(e^{2x})dx\\
	&\int \frac{dw}{\cos^2 w-\cos 2w}
\end{align}