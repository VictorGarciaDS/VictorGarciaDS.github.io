\subsection{Integración por partes}
La tabla \ref{tabla1} posee la fórmula de derivada y de diferencial para el producto $uv$, lo cual es conocido como la \textbf{Regla de Leibniz}. En dicha tabla no aparece una fórmula para la obtención de la integral de un producto, ya que un solo factor, por muy simple que sea, puede complicar bastante la integral.

Consideremos el \cref{ejercicio2.5.1}, $\int \frac{dx}{e^{2x}}$. La aparición del factor $x$ en el numerador, complica enormemente el proceso de integración, pues hasta ahora no tenemos los elementos necesarios para determinar $\int \frac{x\:dx}{e^{2x}}$, más aún, la aparición de factores adicionales en una integral ``sencilla'' o ``que ya sabemos realizar''. Por más pequeño que sea el cambio realizado, puede complicar mucho una integral.

A continuación expondremos la idea detrás de la integración por partes. Comencemos con \cref{eqn:1.3.6},
\begin{align}
	d(uv)&=u\:dv+v\:du\nonumber
	\intertext{De lo anterior tenemos que}
	u\:dv&=d(uv)-v\:du\nonumber
	\intertext{Integrando ambos lados,}
	\int u\:dv&=\int \left(d(uv)-v\:du\right)\nonumber\\
			&=\int d(uv)-\int v\:du\nonumber
			\intertext{Finalmente, por el Teorema Fundamental del Cálculo,}
	\int u\:dv&=uv-\int v\:du\label{eqn:3.3.1}
\end{align}

La fórmula \ref{eqn:3.3.1} es conocida como la \textbf{Fórmula de Integración Por Partes} y su uso tiene como objetivo pasar de la integral $\int u\:dv$ a la integral $\int v\:du$, de manera que esta última sea más sencilla.

Como pudimos ver en los problemas \ref{problema8} y \ref{problema9}, asumir que $du$ está compuesto por el resto de términos de la expresión puede causar errores importantes en el proceso de integración, sin embargo al efectuar una integración por partes, tras elegir $u$, el resto de términos conforman $dv$, sin embargo la elección de $u$ y $dv$ no siempre es evidente.

Además de buscar que $\int v\:du$ sea más sencilla que la integral inicial, para la obtención de $v$ a partir de la elección de $dv$, es necesario que $dv$ se pueda integrar fácilmente, mientras que la obtención de $du$ a partir de $u$ consiste simplemente en diferenciar $u$, lo cual es bastante más sencillo.
\begin{problema}[$\int x\:e^x\:dx$]\label{problema13}
	Supongamos que $u=xe^x$, con lo que $dv=dx$, de esta forma $du=(xe^x+e^x)dx$ y $v=x$. Luego, usando \cref{eqn:3.3.1} de integración por partes, obtenemos que
	\begin{align*}
		\int x\:e^x\:dx&=\int u\:dv\\
					&=uv-\int v\:du\\
					&=xe^x(x)-\int (x)(xe^x+e^x)dx\\
					&=x^2e^x-\int x^2e^x\:dx-\int x\:e^x\:dx.
	\end{align*}
	Sin embargo, la aparición de $\int x^2e^x\:dx$, que es de un grado mayor que la integral inicial, indica que el problema es de mayor dificultad, así que hay que intentar con una elección distinta de $u$.
	
	Supongamos que $u=e^x$, con lo que $dv=x\: dx$, de esta forma $du=e^x\:dx$ y $v=\frac{x^2}{2}$. Luego,
	\begin{align*}
		\int x\:e^x\:dx&=\int u\:dv\\
					&=uv-\int v\:du\\
					&=\frac{x^2}{2}e^x-\int \frac{x^2}{2}e^x\:dx\\
					&=\frac{x^2}{2}e^x-\frac{1}{2}\int x^2\:e^x\:dx,
	\end{align*}
	De nuevo nos encontramos con $\int x^2\:e^x\:dx$, es decir, se presentó la misma dificultad que con la elección de variables anteriores. Tomemos ahora $u=x$ y $dv=e^x\:dx$, así que $du=dx$ y $v=e^x$. Entonces,
	\begin{align*}
		\int x\:e^x\:dx&=\int u\:dv\\
					&=uv-\int v\:du\\
					&=x\:e^x-\int e^x\:dx\\
					&=x\:e^x-e^x+C.
	\end{align*}
\end{problema}
Como pudimos ver en \cref{problema13}, es muy importante que la elección de $u$ permita que $\int v\:du$ sea más sencilla que la integral inicial y no se vuelva más complicada.
\begin{problema}[$\int\:x\:ln\:x$]\label{problema14}
	Una manera de resolver esta integral de manera bastante natural es mediante el cambio de variable $x=e^y$. Intentemos un procedimiento alterno usando el proceso de integración por partes.
	
	La sugerencia anterior y la idea de que $u=x$ redujo el grado de la integral en \cref{problema13}, sugiere la elección de $u=x$, con lo que $dv$ queda determinada como $dv=ln\:x\:dx$, así que $du=dx$ y $v=\int x\:ln\:x\:dx$, que aunque puede resolverse mediante otra integral por partes\footnote{Véase \cref{problema16}.}, por la complejidad que conlleva, por ahora consideraremos como una elección incorrecta, ya que volvimos al problema inicial, así que supondremos $u=ln\:x$, con lo que $dv=x\:dx$, $du=\frac{dx}{x}$ y $v=\frac{x^2}{2}$.
	\begin{align*}
		\int x\:ln\:x\:dx&=\int u\:dv\\
					&=uv-\int v\:du\\
					&=\frac{x^2}{2}\:ln\:x-\int \frac{x^2}{2}\frac{dx}{x}\\
					&=\frac{x^2}{2}\:ln\:x-\frac{1}{2}\int x\:dx\\
					&=\frac{x^2}{2}\:ln\:x-\frac{x^2}{4}+C
	\end{align*}
\end{problema}

Como hemos visto, a diferencia de las derivadas, no existe un recetario bien definido de cuando utilizar un método o una fórmula. A partir del capítulo sobre integración por partes de \cite{simplificadas}, podemos obtener la tabla \ref{tabla3} con ``sugerencias'' respecto a la manera en la que se pueden elegir las variables para pasar de una integral complicada a otra más sencilla:
\begin{center}
	\begin{tabular}{|c|c|}
		\hline
		$u$&$dv$\\
		\hline
		Algebraica&Trigonométrica\\
		Algebraica&Exponencial\\
		Exponencial&Trigonométrica\\
		Logarítmica&$dx$\\
		Logarítmica&Algebraica\\
		Inversa trigonométrica&$dx$\\
		Inversa trigonométrica&Algebraica\\
		\hline
	\end{tabular}\label{tabla3}
\end{center}
Aunque los Problemas \ref{problema13} y \ref{problema14} respaldan la información presentada en la tabla \ref{tabla3}, hay que considerar que es posible encontrarse con integrales que, al intentar resolverse guiados por la tabla anterior, $\int v \:du$ sea más complicada que $\int u\:dv$, o bien, no sea posible obtener explícitamente $v=\int dv$, por lo que no debemos descartar otras posibles eleccciones de $u$ y $dv$.

Es natural pensar que para calcular la integral $\int x^2\:e^x\:dx$, sin conocer el resultado d\cref{problema13}, basta con efectuar el método de integración por partes 2 veces. Más aún, la aplicación iterativa del método de integración por partes, nos permite resolver integrales como la siguiente:
\begin{problema}[$\int x^m e^{x}$, para $m\in\mathbb N\cup\{0\}$]
	A partir de \cref{eqn:2.3.6} y \cref{problema13}, tenemos los primeros 2 casos:
	\begin{equation}\label{eqn:3.3.2}
		\int x^m\: e^x=\left\{\begin{matrix}e^x+C&\text{, si m=0}\\ e^x(x-1)+C&\text{, si m=1}\end{matrix}\right.
	\end{equation}
	Supongamos que $m=2$, tomemos entonces $u=x^2$ y $dv=e^x\:dx$, de manera que $du=2x\:dx$, y de la ecuación \ref{eqn:3.3.2}, tenemos que $v=e^x$, así que
	\begin{align*}
		\int x^2\: e^x&=x^2\:e^x-\int 2x\:e^x\:dx\\
					&=x^2\:e^x-2\int x\:e^x\:dx
		\intertext{De la ecuación \ref{eqn:3.3.2}, tenemos que}
					&=x^2\:e^x-2(xe^x-e^x)+C\\
					&=e^x(x^2-2x+2)+C
	\end{align*}
	Análogamente podemos obtener $\int x^3\: e^x\:dx=e^x(x^3-3x^2+6x-6)+C$.
	De manera general, podemos obtener la siguiente expresión recursiva
	\begin{align*}
		\int x^m\: e^x&=x^me^x-m\int x^{m-1}e^x\:dx
	\end{align*}
	Una expresión explícita de la integral aparece en el ejercicio \ref{ej:3.3.1}.
\end{problema}
En ocasiones, el uso repetido de la fórmula de integración por partes, puede llevar incorrectamente a volver a la integral inicial, pues ocurre lo siguiente
$$\int u\:dv=uv-\int v\:du=uv-\left(vu-\int u\:dv\right)=\int u\:dv$$
Sin embargo, existen integrales específicas en las que aunque parece que se volvió a la integral inicial, se trata de integrales que se concluyen con un despeje, tal es el caso d\cref{problema16}.
\begin{problema}[$\int e^x\:sen\:x\:dx$]\label{problema16}
	Veamos que al tomar $u=e^x$, tenemos que $dv=sen\:x\:dx$, $du=e^x\:dx$ y $v=-cos\:x$, entonces
	\begin{align*}
		\int e^x\:sen\:x\:dx&=-e^x\:cos\:x-\int (-cos\:x)e^x\:dx\\
						&=-e^x\:cos\:x+\int e^x\:cos\:x\:dx
		\intertext{Si tomamos $u$ de manera que $dv=e^x\:dx$, volveríamos a la integral inicial. Entonces tomemos $u=e^x$, con lo que $dv=cos\:x\:dx$, $du=e^x\:dx$ y $v=sen\:x$, entonces}
		\int e^x\:sen\:x\:dx&=-e^x\:cos\:x+e^x\:sen\:x-\int e^x\:sen\:x\:dx
		\intertext{Podemos ver que aunque aparece la misma integral a ambos lados de la igualdad, éstas poseen coeficientes distintos, así que podemos despejar de la siguiente manera, además de agregar la constante de integración:}
		2\int e^x\:sen\:x\:dx&=-e^x\:cos\:x+e^x\:sen\:x\\
		\int e^x\:sen\:x\:dx&=\frac{-e^x\:cos\:x+e^x\:sen\:x}{2}+C\\
						&=e^x\left(\frac{sen\:x-cos\:x}{2}\right)+C
	\end{align*}
\end{problema}
Un problema interesante en el que se combinan los dos artificios de integración estudiados hasta ahora es el siguiente:
\begin{problema}[$\int \sqrt{x^2+1}dx$]
	Usando la tabla \ref{tabla2}, efectuemos la sustitución $x=tan\:\theta$, de manera que $dx=sec^2\:\theta\:d\theta$, entonces
	\begin{align}
		\int \sqrt{x^2+1}dx&=\int \sqrt{tan^2\:\theta+1}sec^2\:\theta\:d\theta\nonumber\\
		&=\int sec\:\theta\:sec^2\:\theta\:d\theta\nonumber\\
		&=\int sec^3\:\theta\:d\theta\label{17a}
	\end{align}
	Veamos que
	\begin{align}
		\int sec^3\theta\:d\theta&=\int sec\:\theta\:sec^2\:\theta\:d\theta\nonumber\\
		&=\int sec\:\theta\:(1+tan^2\:\theta)\:d\theta\nonumber\\
		&=\int sec\:\theta\:d\theta+\int tan^2\theta\:sec\:\theta\:d\theta\nonumber\\
		&=ln(tan\:\theta+sec\:\theta)+\int tan^2\:sec\:\theta\:d\theta\nonumber
		\intertext{Supongamos que $u=tan\theta$, con lo que $dv=tan\:\theta\:sec\:\theta$, $du=sec^2\theta\:d\theta$, $v=sec\:\theta$, así que}
		&=ln(tan\:\theta+sec\:\theta)+tan\:\theta\:sec\:\theta-\int sec^3\:\theta\:d\theta\nonumber\\
		2\int sec^3\theta\:d\theta&=ln(tan\:\theta+sec\:\theta)+tan\:\theta\:sec\:\theta\nonumber\\
		\int sec^3\theta\:d\theta&=\frac{ln(tan\:\theta+sec\:\theta)+tan\:\theta\:sec\:\theta}{2}+C\label{17b}
	\end{align}
	Al sustituir \ref{17b} en \ref{17a}, obtenemos
	\begin{align*}
		\int \sqrt{x^2+1}dx&=\frac{ln(tan\:\theta+sec\:\theta)+tan\:\theta\:sec\:\theta}{2}+C
		\intertext{Ahora, a partir de la tabla \ref{tabla2}, podemos concluir que}
		&=\frac{ln\left(x+\sqrt{x^2+1}\right)+x\sqrt{x^2+1}}{2}+C
	\end{align*}
\end{problema}
\begin{problema}[$\int\frac{e^{3x}\:dx}{\sqrt{1-e^x}}$]
Para resolver esta integral, comencemos resolviendo $\int\frac{e^{x}\:dx}{\sqrt{1-e^x}}$, pues la usaremos varias veces a lo largo de la solución. Tomemos $n=-\frac{1}{2}$ y $v=1-e^x$, con lo que $dv=-e^x\:dx$. Sustituyendo estos valores en \ref{eqn:2.3.4}
	\begin{align}
		\int\frac{e^{x}\:dx}{\sqrt{1-e^x}}&=-\frac{(1-e^x)^{-\frac{1}{2}+1}}{-\frac{1}{2}+1}+C\nonumber\\
									&=-2\sqrt{1-e^x}+C\label{eqn:3.3.5}
	\end{align}
	Volviendo a la integral principal, tomemos $u=e^{2x}$ y $dv=\frac{e^{x}\:dx}{\sqrt{1-e^x}}$, con lo que $du=2e^{2x}\:dx$ y, de acuerdo a \ref{eqn:3.3.5}, $v=-2\sqrt{1-e^x}$, así que usando la fórmula de integración por partes,
	\begin{align*}
		\int\frac{e^{3x}\:dx}{\sqrt{1-e^x}}&=e^{2x}\left(-2\sqrt{1-e^x}\right)-\int \left(-2\sqrt{1-e^x}\right)2e^{2x}\:dx\\
		&=-2e^{2x}\sqrt{1-e^x}+4\int e^{2x}\sqrt{1-e^x}\:dx\\
		&=-2e^{2x}\sqrt{1-e^x}+4\int e^{2x}\frac{1-e^x}{\sqrt{1-e^x}}\:dx\\
		&=-2e^{2x}\sqrt{1-e^x}+4\int \frac{e^{2x}}{\sqrt{1-e^x}}\:dx-4\int \frac{e^{3x}}{\sqrt{1-e^x}}\:dx
	\end{align*}
	Así que
	\begin{align}
		\int\frac{e^{3x}\:dx}{\sqrt{1-e^x}}+4\int\frac{e^{3x}\:dx}{\sqrt{1-e^x}}&=-2e^{2x}\sqrt{1-e^x}+4\int \frac{e^{2x}}{\sqrt{1-e^x}}\:dx\nonumber\\
		5\int\frac{e^{3x}\:dx}{\sqrt{1-e^x}}&=-2e^{2x}\sqrt{1-e^x}+4\int \frac{e^{2x}}{\sqrt{1-e^x}}\:dx\label{eqn:3.3.6}
	\end{align}
Tomemos ahora $u=e^x$ y $dv=\frac{e^{x}\:dx}{\sqrt{1-e^x}}$, con lo que $du=e^x\:dx$ y $v=-2\sqrt{1-e^x}$, así que
\begin{align*}
		&=-2e^{2x}\sqrt{1-e^x}+4\left[(e^x)(-2\sqrt{1-e^x})-\int (-2\sqrt{1-e^x})e^x\:dx\right]\\
		&=-2e^{2x}\sqrt{1-e^x}-8e^x\sqrt{1-e^x}+8\int e^x\sqrt{1-e^x}\:dx\\
		&=-2e^{2x}\sqrt{1-e^x}-8e^x\sqrt{1-e^x}+8\int e^x\frac{1-e^x}{\sqrt{1-e^x}}\:dx\\
		&=-2e^{2x}\sqrt{1-e^x}-8e^x\sqrt{1-e^x}+8\int \frac{e^x}{\sqrt{1-e^x}}\:dx-2\left(4\int \frac{e^{2x}}{\sqrt{1-e^{x}}}\:dx\right)
		\intertext{De las ecuaciones \ref{eqn:3.3.5} y \ref{eqn:3.3.6}, tenemos que}
		&=-2e^{2x}\sqrt{1-e^x}-8e^x\sqrt{1-e^x}+8\left(-2\sqrt{1-e^x}\right)-2\left(5\int\frac{e^{3x}\:dx}{\sqrt{1-e^x}}+2e^{2x}\sqrt{1-e^x}\right)\\
		&=-2e^{2x}\sqrt{1-e^x}-8e^x\sqrt{1-e^x}-16\sqrt{1-e^x}-10\int\frac{e^{3x}\:dx}{\sqrt{1-e^x}}-4e^{2x}\sqrt{1-e^x}\\
		&=-6e^{2x}\sqrt{1-e^x}-8e^x\sqrt{1-e^x}-16\sqrt{1-e^x}-10\int\frac{e^{3x}\:dx}{\sqrt{1-e^x}}
	\end{align*}
	Entonces,
	\begin{align*}
		5\int\frac{e^{3x}\:dx}{\sqrt{1-e^x}}+10\int\frac{e^{3x}\:dx}{\sqrt{1-e^x}}&=-6e^{2x}\sqrt{1-e^x}-8e^x\sqrt{1-e^x}-16\sqrt{1-e^x}
		\intertext{Así que}
		15\int\frac{e^{3x}\:dx}{\sqrt{1-e^x}}&=-6e^{2x}\sqrt{1-e^x}-8e^x\sqrt{1-e^x}-16\sqrt{1-e^x}
		\intertext{Finalmente,}
		\int\frac{e^{3x}\:dx}{\sqrt{1-e^x}}&=-\frac{6}{15}e^{2x}\sqrt{1-e^x}-\frac{8}{15}e^x\sqrt{1-e^x}-\frac{16}{15}\sqrt{1-e^x}+C\\
		&=-\frac{2}{15}\sqrt{1-e^x}\left(3e^{2x}+4e^x+8\right)+C
	\end{align*}
\end{problema}

\subsubsection{Ejercicios}
\begin{align}
	\int x^m\:e^x\:dx&=e^x\left(\sum_{i=0}^m (-1)^i\frac{m!}{(m-i)!}x^{m-i}\right)+C\text{ para }m\in \mathbb N\cup\{0\}\label{ej:3.3.1}
\end{align}

\subsubsection{Problemas}
\begin{align}
	\int cos^2\:y\:dy\label{problema3.3.2.1}
\end{align}