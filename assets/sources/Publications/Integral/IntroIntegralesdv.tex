\subsection{La importancia de la elección de $dv$}\label{subsec3.2}
Un error muy común cuando se está aprendiendo a integrar, consiste en creer que tras elegir $v$, el resto de términos conforman $dv$, para notar que no es así, basta con observar que para el siguiente problema, derivado del \cref{problema5}, es fácil cometer un error.
\begin{problema}[$\int 5\: ln\:a\: a^x\:dx$]\label{problema6}
	Al tomar $v=a^x$, no debemos afirmar tan a la ligera que $dv$ es $5\:ln\:a\:dx$, pues $dv$ se determina al \textbf{diferenciar} $v$, asistidos de las fórmulas de la sección \ref{subsec:2.1.1}, no simplemente tomando el resto de términos en la expresión. Como se vio en el \cref{problema5}, $dv=ln\:a\:a^x$, así que no es posible usar directamente \cref{eqn:2.3.1} con $v(x)=a^x$, así como en el \cref{problema5} tampoco se usó desde el primer paso.
	
	Comencemos entonces usando \cref{eqn:2.3.2} con $c=5$ y $v=ln\:a\:a^x$, de manera que
	\begin{align*}
		\int 5\: ln\:a\: a^x\:dx&=5\int ln\:a\: a^x\:dx
		\intertext{Donde ya podemos usar \cref{eqn:2.3.1} con $v(x)=a^x$}
								&=5a^x+C
	\end{align*}
\end{problema}
En ocasiones una constante será lo único que impida usar directamente alguna fórmula y como se vio en el \cref{problema6}, la manera de tratar con este impedimento es asistidos de \cref{eqn:2.3.2}, sin embargo no siempre será así y serán requeridas otras estrategias, como las expuestas en la \cref{sec4}.

Una buena estrategia para asegurarse de no usar incorrectamente una fórmula es cubrir la integral que se está realizando para determinar $dv$ a partir de la elección de $v$ y no a partir de la integral que se está resolviendo.