\subsection{Algunas integrales casi inmediatas}
A continuación veremos otras integrales que requieren de un esfuerzo ligeramente mayor a la simple observación de la fórmula y la correcta elección de la función $v(x)$ a integrar.
\begin{problema}[$\int tan\:v\:dv$]\label{problema1}
Comencemos utilizando la siguiente identidad trigonométrica:
\begin{align}\label{eqn:2.4.1}
	\int tan\:x\:dx&=\int \frac{sen\:x}{cos\:x}\:dx\nonumber\\
	&=-\int \frac{-sen\:x\:dx}{cos\:x}\nonumber
\intertext{Podemos ver que al sustituir $v=x$ en \cref{eqn:2.2.2} obtenemos el numerador de la expresión anterior, así que podemos reescribir lo anterior como}
				&=-\int \frac{d(cos\:x)}{cos\:x}\nonumber
			\intertext{Al usar \cref{eqn:2.3.5} con $v(x)=cos\:x$, obtenemos}
				&=-ln(cos\:x)+C\nonumber\\
	\int tan\:v\:dv&=-ln(cos\:v)+C
\end{align}
\end{problema}
\begin{problema}[$\int cot\:v\:dv$]\label{problema2}
La demostración es análoga a la integral \ref{eqn:2.4.1}
\begin{align}
\int cot\:v\:dv=ln(sen\:v)+C
\end{align}
\end{problema}
\begin{problema}[$\int sec\:v\:dv$]\label{problema3}
Al observar las fórmulas \ref{eqn:2.2.1} a \ref{eqn:2.2.6}, podemos notar que no hay expresión trigonométrica, cuya diferencial sea $sec\:v\:dv$, sin embargo, \cref{eqn:2.2.3,eqn:2.2.5} tienen expresiones parecidas, así que multiplicamos y dividimos la expresión por $sec\:x+tan\:x$:
\begin{align}
\int sec\:x\:dx&=\int \frac{sec\:x+tan\:x}{sec\:x+tan\:x} sec\:x\:dx\nonumber\\
			&=\int \frac{(sec\:x+tan\:x)sec\:x\:dx}{sec\:x+tan\:x}\nonumber\\
			&=\int \frac{(sec^2\:x+tan\:x\:sec\:x)\:dx}{sec\:x+tan\:x}\nonumber
			\intertext{Al reordenar los sumandos del denominador}
			&=\int \frac{(sec^2\:x+tan\:x\:sec\:x)\:dx}{tan\:x+sec\:x},\nonumber
			\intertext{Al sustituir en \cref{eqn:1.3.4} $u=tan\:x$, $v=sec\:x$ y $w=0$, además de \cref{eqn:2.2.3,eqn:2.2.5}, antes mencionadas, ambas con $v=x$, podemos reescribir el numerador de la forma}
			&=\int \frac{d(tan\:x+\:sec\:x)}{tan\:x+sec\:x}\nonumber\\
			\intertext{Finalmente al usar \cref{eqn:2.3.5} con $v(x)=tan\:x+sec\:x$, obtenemos}
			&=ln(tan\:x+sec\:x)+C\nonumber\\
\label{eqn:2.4.3}
\int sec\:v\:dv&=ln(tan\:v+sec\:v)+C
\end{align}
\end{problema}
\begin{problema}[$\int csc\:v\:dv$]
La demostración es análoga a la integral \ref{eqn:2.4.3}
\begin{align}
\int csc\:v\:dv=-ln(cot\:v+csc\:v)+C
\end{align}
\end{problema}
\begin{problema}[$\int a^v\:dv$]\label{problema5}
El caso $a=1$ se ve en \cref{eqn:2.3.1}, así que supondremos $a\neq 1$.
Multiplicamos y dividimos por $\ln a$,
\begin{align}
	\int a^x\:dx&=\int \frac{\ln a}{\ln a} a^x\:dx\nonumber\\
			&=\int \frac{1}{\ln a} \ln a\cdot a^x\:dx\nonumber
\intertext{Al utilizar \cref{eqn:2.3.2} con $c=\frac{1}{\ln a}$ y $v=\ln a\cdot a^x$, obtenemos}
			&=\frac{1}{\ln a}\int \ln a\cdot a^x\:dx\nonumber
\intertext{Al usar \cref{eqn:2.2.16} con $v=x$, obtenemos}
			&=\frac{1}{\ln a}\int d(a^x)\nonumber
\intertext{Así que usar el Teorema Fundamental del Cálculo, antes enunciado como la versión general de \cref{eqn:2.3.1} con $v(x)=a^x$, concluimos que}
			&=\frac{1}{\ln a}a^x+C\nonumber\\
	\int a^v\:dv&=\frac{1}{\ln a}a^v+C
\end{align}
\end{problema}